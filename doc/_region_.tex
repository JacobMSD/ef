\message{ !name(ef.tex)}\documentclass[12pt,a4paper]{article}
% Page parameters
\usepackage[
left=3cm,
right=2cm,
top=2cm,
bottom=2cm,
bindingoffset=0cm]{geometry}
% Lines interval
\renewcommand\baselinestretch{1.25}
% Fonts
%\usepackage{cmlgc}
%
\usepackage[T2A]{fontenc}
\usepackage[utf8]{inputenc}
%\usepackage{biblatex}
\usepackage{indentfirst}
\usepackage{amssymb,amsmath}
\usepackage{empheq}
\usepackage{graphicx}
\usepackage{enumerate}
\usepackage{hyperref}
%\usepackage{appendix}
%\usepackage{doi}
\usepackage[space]{cite}
%

% Colors
%\usepackage{color}
%\usepackage[usenames,dvipsnames,svgnames,table]{xcolor}

% Custom commands
\usepackage{common_math}
\newcommand{\todo}[1]{\textcolor{red}{#1}}

% Section style (Dot after arabic numer).
%\usepackage{titlesec}
%\titlelabel{\thetitle.\quad}

% Dots in Table Of contents.
%\usepackage[subfigure]{tocloft}
%\renewcommand{\cftsecleader}{\cftdotfill{\cftdotsep}}

% Equation numeration
\numberwithin{equation}{section}
%\numberwithin{equation}{subsection}
%\numberwithin{equation}{subsubsection}

\title{}
\begin{document}

\message{ !name(ex4_conducting_sphere_potential.tex) !offset(-52) }
\section{Conduction sphere potential}
Currently epicf is not capable to load any external electic fields or
use any analytic expressions for them.  However, it is possible to
specify geometric primitives ( such as sphere, cube, etc. ) inside
computation domain, that will act like conductors with fixed
potential. To some extent, this allows to produce desired field
destribution.

In this example, numerically calculated potential of charged
conducting sphere is compared with analytical expression.

Potential of a charged conducting sphere with radius R is constant
inside the sphere and equals to point charge potential outside the
sphere:
\begin{gather}
  \varphi( \mathbf{r} )
  = 
  \begin{cases}
    \varphi_0, \,\, r \le R \\
    \dfrac{R}{r} \varphi_0, \,\, r > R
  \end{cases}
\end{gather}

In config file, definition of such sphere is done inside
\texttt{[Inner_region_sphere]} section.

\begin{verbatim}
    [Inner_region_sphere.sphere]
    potential = 3.0
    sphere_origin_x = 5.0
    sphere_origin_y = 5.0
    sphere_origin_z = 5.0
    sphere_radius = 0.5
\end{verbatim}

Since we don't want any particles, there is no need to specify sources. 

Currently, there is no option in Epicf to calculate fields only - the program will attempt to evaluate several dynamics time steps. To mitigate such effect, it is possible to set saving time step greter than total simulation time:

[code]

When analysing results, it is possible to plot 2d potential distribution similar to previous example.
[pics]

More illustrative is plot along axis passing though a center of the sphere:
<img src="https://github.com/epicf/epicf/blob/dev/doc/figs/ex4_conducting_sphere_potential/potential_along_z.png" width="500"/>

It can be seen, that agreement is not very good. This can be related to the fact, that 
in the simulation potentials on boundaries of computation domain are
set to zero. In reality potential potential slowly approaches zero as
distance from the sphere increases.  

%%% Local Variables:
%%% mode: latex
%%% TeX-master: "ef"
%%% End:

\message{ !name(ef.tex) !offset(-50) }

\end{document}
