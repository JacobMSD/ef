\section{Introduction}

Ef is a software for charged particles dynamics simulation.  It's
primary areas of application are accelerator science and plasma
physics.

Ef focuses on nonrelativistic energies. Currently, particular
emphasis is placed on low-energy beams, such that can be found in ion
sources and electron guns.

Particles dynamics is traced under action of external electromagnetic
fields.  Particle self-interaction is taken into account with
\href{https://en.wikipedia.org/wiki/Particle-in-cell}{particle-in-cell} 
method.

Attention is given to integration with CAD software to allow for
simulation of complex real-life experimental setups.  An experimental
plugin for \href{http://www.freecadweb.org/}{FreeCAD} exists.

Ef is a free software -- it's source code is open and avaible for
modification and redistribution.  It is written mostly in C++.  Basic
MPI support allows to take advantage of parallel execution in
multiprocessor environment.

\href{https://github.com/epicf/epicf/wiki/1-Rationale-and-Aims}{Development goals}
and 
\href{https://github.com/epicf/epicf/wiki/3-Feature-Matrix-and-Development-Roadmap}{current features}
are described in detail in appropriate wiki sections,
as well as \href{https://github.com/epicf/epicf/wiki/4-Installation}{installation procedure}.
Some usage \href{https://github.com/epicf/epicf/wiki/Examples}{examples}
are also given.

%%% Local Variables:
%%% mode: latex
%%% TeX-master: "ef"
%%% End:
